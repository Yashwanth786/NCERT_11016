\let\negmedspace\undefined
\let\negthickspace\undefined
\documentclass[journal,12pt,twocolumn]{IEEEtran}
\usepackage{cite}
\usepackage{amsmath,amssymb,amsfonts,amsthm}
\usepackage{algorithmic}
\usepackage{graphicx}
\usepackage{textcomp}
\usepackage{xcolor}
\usepackage{txfonts}
\usepackage{listings}
\usepackage{enumitem}
\usepackage{mathtools}
\usepackage{gensymb}
\usepackage[breaklinks=true]{hyperref}
\usepackage{tkz-euclide} 
\usepackage{listings}
\newtheorem{theorem}{Theorem}[section]
\newtheorem{problem}{Problem}
\newtheorem{proposition}{Proposition}[section]
\newtheorem{lemma}{Lemma}[section]
\newtheorem{corollary}[theorem]{Corollary}
\newtheorem{example}{Example}[section]
\newtheorem{definition}[problem]{Definition}
\newcommand{\BEQA}{\begin{eqnarray}}
\newcommand{\EEQA}{\end{eqnarray}}
\newcommand{\define}{\stackrel{\triangle}{=}}
\theoremstyle{remark}
\newtheorem{rem}{Remark}

%\bibliographystyle{ieeetr}
\begin{document}
%

\bibliographystyle{IEEEtran}


\vspace{3cm}

\title{
%	\logo{
NCERT-2
%	}
}
\author{EE22BTECH11016 - Ch.Yashwanth}


\maketitle

\newpage

%\tableofcontents

\bigskip

\renewcommand{\thefigure}{\theenumi}
\renewcommand{\thetable}{\theenumi}


\providecommand{\pr}[1]{\ensuremath{\Pr\left(#1\right)}}
\providecommand{\prt}[2]{\ensuremath{p_{#1}^{\left(#2\right)} }}        % own macro for this question
\providecommand{\qfunc}[1]{\ensuremath{Q\left(#1\right)}}
\providecommand{\sbrak}[1]{\ensuremath{{}\left[#1\right]}}
\providecommand{\lsbrak}[1]{\ensuremath{{}\left[#1\right.}}
\providecommand{\rsbrak}[1]{\ensuremath{{}\left.#1\right]}}
\providecommand{\brak}[1]{\ensuremath{\left(#1\right)}}
\providecommand{\lbrak}[1]{\ensuremath{\left(#1\right.}}
\providecommand{\rbrak}[1]{\ensuremath{\left.#1\right)}}
\providecommand{\cbrak}[1]{\ensuremath{\left\{#1\right\}}}
\providecommand{\lcbrak}[1]{\ensuremath{\left\{#1\right.}}
\providecommand{\rcbrak}[1]{\ensuremath{\left.#1\right\}}}
\newcommand{\sgn}{\mathop{\mathrm{sgn}}}
\providecommand{\abs}[1]{\left\vert#1\right\vert}
\providecommand{\res}[1]{\Res\displaylimits_{#1}} 
\providecommand{\norm}[1]{\left\lVert#1\right\rVert}
%\providecommand{\norm}[1]{\lVert#1\rVert}
\providecommand{\mtx}[1]{\mathbf{#1}}
\providecommand{\mean}[1]{E\left[ #1 \right]}
\providecommand{\cond}[2]{#1\middle|#2}
\providecommand{\fourier}{\overset{\mathcal{F}}{ \rightleftharpoons}}
\newenvironment{amatrix}[1]{%
  \left(\begin{array}{@{}*{#1}{c}|c@{}}
}{%
  \end{array}\right)
}

\newcommand{\solution}{\noindent \textbf{Solution: }}
\newcommand{\cosec}{\,\text{cosec}\,}
\providecommand{\dec}[2]{\ensuremath{\overset{#1}{\underset{#2}{\gtrless}}}}
\newcommand{\myvec}[1]{\ensuremath{\begin{pmatrix}#1\end{pmatrix}}}
\newcommand{\mydet}[1]{\ensuremath{\begin{vmatrix}#1\end{vmatrix}}}
\newcommand{\myaugvec}[2]{\ensuremath{\begin{amatrix}{#1}#2\end{amatrix}}}
\providecommand{\rank}{\text{rank}}
\providecommand{\pr}[1]{\ensuremath{\Pr\left(#1\right)}}
\providecommand{\qfunc}[1]{\ensuremath{Q\left(#1\right)}}
	\newcommand*{\permcomb}[4][0mu]{{{}^{#3}\mkern#1#2_{#4}}}
\newcommand*{\perm}[1][-3mu]{\permcomb[#1]{P}}
\newcommand*{\comb}[1][-1mu]{\permcomb[#1]{C}}
\providecommand{\qfunc}[1]{\ensuremath{Q\left(#1\right)}}
\providecommand{\gauss}[2]{\mathcal{N}\ensuremath{\left(#1,#2\right)}}
\providecommand{\diff}[2]{\ensuremath{\frac{d{#1}}{d{#2}}}}
\providecommand{\myceil}[1]{\left \lceil #1 \right \rceil }
\newcommand\figref{Fig.~\ref}
\newcommand\tabref{Table~\ref}
\newcommand{\sinc}{\,\text{sinc}\,}
\newcommand{\rect}{\,\text{rect}\,}
\let\vec\mathbf

\textbf{Question 12.13.6.11}
In a game, a man wins a rupee for a six and loses a rupee for any other number
when a fair die is thrown. The man decided to throw a die thrice but to quit as
and when he gets a six. Find the expected value of the amount he wins / loses.\\
\solution
To solve this problem using the binomial distribution, we can first consider each throw of the die as a Bernoulli trial. In each trial, the man either wins a rupee (when he rolls a six) or loses a rupee (for any other number). Let's define the random variable:

\[
X = 
\begin{cases}
1 & \text{if he wins a rupee (rolls a six)} \\
-1 & \text{if he loses a rupee (other than six)}
\end{cases}
\]

The probability of winning a rupee (rolling a six) in a single trial is denoted by $p$, and the probability of losing a rupee (rolling any other number) is denoted by $q$. Since it's a fair die, $$p = \frac{1}{6}\quad and \quad q = \frac{5}{6}$$

Now, the man will continue to throw the die until he rolls a six. This can take 1, 2, or 3 throws (because he decides to quit as soon as he gets a six). We can calculate the expected value of the amount he wins/loses for each of these cases:
$$ \quad $$

1. If he wins on the first throw, then he gets one rupee so overall gain = 1 (probability of success, $p$):
\[
E(X_1) = 1 \cdot p = \frac{1}{6}
\]

2. If he wins on the second throw, then he loses a rupee in first throw and gains a rupee in second throw so overall gain = 0 (probability of failure on the first throw, $q$, and then success on the second throw, $p$):
\[
E(X_2) = (-1 +1) \cdot q \cdot p = 0
\]

3. If he wins on the third throw, then he loses two rupees in first two throws and gain a rupee in third throw so overall gain = -1 (probability of failure on the first two throws, $q^2$, and then success on the third throw, $p$):
\[
E(X_3) = (-1 -1 +1) \cdot q^2 \cdot p = -\left(\frac{5}{6}\right) \cdot \left(\frac{5}{6}\right) \cdot \left(\frac{1}{6}\right) = -\frac{25}{216}
\]

4. If he loses in all three throws, then he loses three rupees so overall gain = -3 (probability of failure on the first two throws, $q^3$):
\[
E(X_4) = (-1 -1 -1) \cdot q^3 = -3 \cdot \left(\frac{5}{6}\right) \cdot \left(\frac{5}{6}\right) \cdot \left(\frac{5}{6}\right) = -\frac{375}{216}
\]

Now, let's calculate the expected value of the amount he wins/loses in total:

Overall Expected Value ($E$) = $E(X_1) + E(X_2) + E(X_3) + E(X_4)$
$$
E = \frac{1}{6} + 0 + \left(-\frac{25}{216}\right) + \left(-\frac{375}{216}\right)
  \approx -1.685
$$

So, the expected value of the amount he wins/loses is approximately -1.685 rupees. This means, on average, he is expected to lose about 1.685 rupees when playing this game.

\end{document}
