\let\negmedspace\undefined
\let\negthickspace\undefined
\documentclass[journal,12pt,twocolumn]{IEEEtran}
\usepackage{cite}
\usepackage{amsmath,amssymb,amsfonts,amsthm}
\usepackage{algorithmic}
\usepackage{graphicx}
\usepackage{textcomp}
\usepackage{xcolor}
\usepackage{txfonts}
\usepackage{listings}
\usepackage{enumitem}
\usepackage{mathtools}
\usepackage{gensymb}
\usepackage[breaklinks=true]{hyperref}
\usepackage{tkz-euclide} 
\usepackage{listings}
\newtheorem{theorem}{Theorem}[section]
\newtheorem{problem}{Problem}
\newtheorem{proposition}{Proposition}[section]
\newtheorem{lemma}{Lemma}[section]
\newtheorem{corollary}[theorem]{Corollary}
\newtheorem{example}{Example}[section]
\newtheorem{definition}[problem]{Definition}
\newcommand{\BEQA}{\begin{eqnarray}}
\newcommand{\EEQA}{\end{eqnarray}}
\newcommand{\define}{\stackrel{\triangle}{=}}
\theoremstyle{remark}
\newtheorem{rem}{Remark}

%\bibliographystyle{ieeetr}
\begin{document}
%

\bibliographystyle{IEEEtran}


\vspace{3cm}

\title{
%	\logo{
NCERT-2
%	}
}
\author{EE22BTECH11016 - Ch.Yashwanth}


\maketitle

\newpage

%\tableofcontents

\bigskip

\renewcommand{\thefigure}{\theenumi}
\renewcommand{\thetable}{\theenumi}


\providecommand{\pr}[1]{\ensuremath{\Pr\left(#1\right)}}
\providecommand{\prt}[2]{\ensuremath{p_{#1}^{\left(#2\right)} }}        % own macro for this question
\providecommand{\qfunc}[1]{\ensuremath{Q\left(#1\right)}}
\providecommand{\sbrak}[1]{\ensuremath{{}\left[#1\right]}}
\providecommand{\lsbrak}[1]{\ensuremath{{}\left[#1\right.}}
\providecommand{\rsbrak}[1]{\ensuremath{{}\left.#1\right]}}
\providecommand{\brak}[1]{\ensuremath{\left(#1\right)}}
\providecommand{\lbrak}[1]{\ensuremath{\left(#1\right.}}
\providecommand{\rbrak}[1]{\ensuremath{\left.#1\right)}}
\providecommand{\cbrak}[1]{\ensuremath{\left\{#1\right\}}}
\providecommand{\lcbrak}[1]{\ensuremath{\left\{#1\right.}}
\providecommand{\rcbrak}[1]{\ensuremath{\left.#1\right\}}}
\newcommand{\sgn}{\mathop{\mathrm{sgn}}}
\providecommand{\abs}[1]{\left\vert#1\right\vert}
\providecommand{\res}[1]{\Res\displaylimits_{#1}} 
\providecommand{\norm}[1]{\left\lVert#1\right\rVert}
%\providecommand{\norm}[1]{\lVert#1\rVert}
\providecommand{\mtx}[1]{\mathbf{#1}}
\providecommand{\mean}[1]{E\left[ #1 \right]}
\providecommand{\cond}[2]{#1\middle|#2}
\providecommand{\fourier}{\overset{\mathcal{F}}{ \rightleftharpoons}}
\newenvironment{amatrix}[1]{%
  \left(\begin{array}{@{}*{#1}{c}|c@{}}
}{%
  \end{array}\right)
}

\newcommand{\solution}{\noindent \textbf{Solution: }}
\newcommand{\cosec}{\,\text{cosec}\,}
\providecommand{\dec}[2]{\ensuremath{\overset{#1}{\underset{#2}{\gtrless}}}}
\newcommand{\myvec}[1]{\ensuremath{\begin{pmatrix}#1\end{pmatrix}}}
\newcommand{\mydet}[1]{\ensuremath{\begin{vmatrix}#1\end{vmatrix}}}
\newcommand{\myaugvec}[2]{\ensuremath{\begin{amatrix}{#1}#2\end{amatrix}}}
\providecommand{\rank}{\text{rank}}
\providecommand{\pr}[1]{\ensuremath{\Pr\left(#1\right)}}
\providecommand{\qfunc}[1]{\ensuremath{Q\left(#1\right)}}
	\newcommand*{\permcomb}[4][0mu]{{{}^{#3}\mkern#1#2_{#4}}}
\newcommand*{\perm}[1][-3mu]{\permcomb[#1]{P}}
\newcommand*{\comb}[1][-1mu]{\permcomb[#1]{C}}
\providecommand{\qfunc}[1]{\ensuremath{Q\left(#1\right)}}
\providecommand{\gauss}[2]{\mathcal{N}\ensuremath{\left(#1,#2\right)}}
\providecommand{\diff}[2]{\ensuremath{\frac{d{#1}}{d{#2}}}}
\providecommand{\myceil}[1]{\left \lceil #1 \right \rceil }
\newcommand\figref{Fig.~\ref}
\newcommand\tabref{Table~\ref}
\newcommand{\sinc}{\,\text{sinc}\,}
\newcommand{\rect}{\,\text{rect}\,}
\let\vec\mathbf

\textbf{Question 12.13.6.11}
In a game, a man wins a rupee for a six and loses a rupee for any other number
when a fair die is thrown. The man decided to throw a die thrice but to quit as
and when he gets a six. Find the expected value of the amount he wins / loses.\\
\solution
In this game, a man wins 1 rupee if he gets a six when throwing a fair six-sided die, and loses 1 rupee for any other number. The man has decided to throw the die thrice but to quit as soon as he gets a six. We want to find the expected value of the amount he wins or loses.

Let's analyze the possibilities:

\begin{enumerate}
    \item He gets a six on the first throw: In this case, he wins 1 rupee.
    \item He doesn't get a six on the first throw but gets a six on the second throw: He neither wins nor loses any money. So overall amount is zero
    \item He doesn't get a six on the first two throws but gets a six on the third throw: He loses 2 rupees from the first two throws but wins 1 rupee on the third throw, resulting in a net loss of 1 rupee.
    \item He doesn't get a six in three throws: He loses 3 rupees.
\end{enumerate}

Now, let's calculate the probabilities of these outcomes:
\begin{align*}
    & \text{Probability of getting a six on the first throw =}\quad\frac{1}{6} \\
    & \text{Probability of not getting a six on the first}\\
    & \text{\quad throw and getting a six on the second} \\
    & \text{\quad throw =}\quad\left(\frac{5}{6}\right) \times \left(\frac{1}{6}\right)=\frac{5}{36} \\
    & \text{Probability of not getting a six on the first}\\ 
    & \text{\quad two throws and getting a six on the third}\\
    & \text{\quad throw =}\quad\left(\frac{5}{6}\right) \times \left(\frac{5}{6}\right) \times \left(\frac{1}{6}\right) = \frac{25}{216} \\
    & \text{Probability of not getting a six in three}\\
    & \text{\quad throws =}\quad\left(\frac{5}{6}\right)^3 = \frac{125}{216}
\end{align*}

Now, let's calculate the expected value:

\begin{align*}
    \text{Expected value} 
     & = \left( 1 \times \frac{1}{6}\right) + \left( 0 \times \frac{5}{36}\right) + \left(-1 \times \frac{25}{216}\right)\\
     & \quad \quad + \left(-3 \times \frac{125}{216}\right)\\
     & = \frac{1}{6} - 0 + \left(-\frac{25}{216}\right) - \frac{375}{216}\\
     & = \frac{36}{216} - 0 + \left(-\frac{25}{216}\right) - \frac{375}{216}\\
     & = \frac{36 - 0 - 25 - 375}{216}\\
     & = -\frac{364}{216}\\
     & \approx -1.685
\end{align*}
So, the corrected expected value of the amount the man wins or loses is approximately -1.685 rupees. This means that on average, he can expect to lose about 1.685 rupees per game in the long run.



\end{document}
